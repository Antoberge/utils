\documentclass[12pt]{article} 
\renewcommand{\baselinestretch}{1.3} 
\usepackage[round,authoryear]{natbib}
\usepackage[paper=a4paper,textheight=24cm,textwidth=17cm]{geometry}
\usepackage{amsmath, amstext, amssymb, dsfont}
\usepackage{textcomp}
\usepackage[sc]{mathpazo}
\usepackage[LGR,T1]{fontenc}
\usepackage{graphicx}
\usepackage{bbm, bm}
\usepackage{enumerate}
\usepackage{tabularx}
\usepackage{threeparttable}
\usepackage[scaled]{helvet} 
\usepackage{courier}
\newcolumntype{P}[1]{>{\centering\arraybackslash}p{#1}}
\usepackage{lscape}
\setlength{\parskip}{1ex plus 0.3ex minus 0.3ex} 
\setlength{\skip\footins}{1cm} 
\usepackage{enumitem}
 \usepackage[hang,flushmargin]{footmisc}
\usepackage{footnote}
\usepackage{multirow}
\usepackage{array}
\usepackage{lipsum}  
%\usepackage{xcolor}
\usepackage[usenames,dvipsnames]{color}
\usepackage{rotating}
\usepackage[utf8]{inputenc}
\usepackage{adjustbox}
\usepackage{datetime}
\usepackage{amsthm}
\setlength{\parindent}{0cm}
\usepackage{tikz}
\usepackage{fontawesome5}
\usepackage{relsize}
\usetikzlibrary{shapes.geometric, arrows, trees, positioning, fit, decorations.pathreplacing}
\definecolor{citecolor}{rgb}{.64,0,0}
\definecolor{bgreen}{rgb}{0.05, 0.5, 0.06}
\definecolor{bgfond}{RGB}{115, 106, 232}
\usepackage{longtable}
\usepackage[colorlinks=true, linkcolor=red, citecolor = citecolor, urlcolor = blue]{hyperref}
\usepackage{bookmark}
\usepackage{appendix}
\usepackage{tabularx}
\usepackage{booktabs}
\usepackage{caption}
\usepackage{siunitx}
\usepackage{setspace}
\usepackage{comment}
\usepackage{float}
\usepackage{subfig}
\usepackage[section]{placeins}

\begin{document}

\title{\textbf{Research Statement}}
\author{Antonin Bergeaud}
\date{Dec. 2023}

\maketitle

Technological advancements and innovation are essential drivers of economic growth and prosperity. At the same time, they also present significant questions for our future. My research sits at the crossroads of innovation and macroeconomics, aiming to enhance our understanding of how technological change impacts firm performance, growth, the future of work, inequality, and international trade.

\section*{Methodology and Research Orientation}

The pioneering endogenous growth models, created more than 30 years ago, have paved the way to a new stream of research on the determinants of long-term economic development. This research has subsequently gained renewed momentum driven by the availability of micro-data that allows for direct testing of underlying mechanisms and by new theoretical insights that highlight the role of firm heterogeneity in explaining aggregate facts. In my research, I aim to blend theoretical models with empirical data analysis. I believe this iterative process between theory and empirical evidence, where each informs the other, is essential to foster the development of novel ideas and to help identify and evaluate the inefficiencies and frictions that impede long-term growth.

Measuring and understanding the role of innovation and technology in economics is a challenging endeavor that necessitates the development of new tools and strategies. In my work, I devise new datasets by merging existing data sources (like patents) with novel statistical tools (such as machine learning and natural language processing) to address this challenge. I am a firm advocate for making such data as accessible as possible to the community, ensuring they are available as soon as they are prepared.

Moreover, I believe that historical insights can guide current policy debates on innovation and industrial strategy, helping to derive more accurate predictions about the future impact of emerging technologies like AI. Both in my past and forthcoming work, I endeavor to leverage historical data to reexamine previous technological shifts and assess their influence on growth and development. 


\section*{Long-run productivity trends}

My first line of research consists of a series of studies about the long-run determinant of productivity trends and innovation. A major source of concerns in our modern economies is the slowdown observed since the 1990s in Europe and the United States, despite significant technological advances and numerous policies aimed at revitalizing productivity and the competitiveness of companies. To place this slowdown in a historical perspective, we created a new database in \citet{bergeaud2016productivity}. This database
measures and compares the level of Total Factor Productivity (TFP) to statistically highlight the periods of disruption. The process involved assimilating yearly data on various input factors and GDP from diverse sources and to make several assumptions in order to deduce the level of Total Factor Productivity. This dataset, consistently updated, can be accessed publicly at \href{http://longtermproductivity.com}{the long term productivity project website}. It has become a widely used data tool by scholars interested in measuring TFP in the long run. In \citet*{bergeaud2016productivity} and subsequently in \citet*{bergeaud2017total} and \citet*{bergeaud2018role}, we confirmed that the observed slowdown appears to be structural.

Many explanations have been proposed in the literature to explain this slowdown as summarized in \citet{bergeaud2019market}. In \citet*{aghion2019missing}, we explore the possibility that this slowdown might result from measurement errors related to the difficulty of integrating the price evolution of consumer goods when they are frequently renewed. We show that this explanation is not sufficient to explain the extent of the observed productivity slowdown in either the United States or France, as subsequently demonstrated in \citet*{aghion2018firm}. 

The slowdown could also have a technological explanation. Numerous pieces of evidence point towards an insufficient adoption of digital technologies by many firms, especially smaller ones, as we document in \citet*{ecb}. This lag in adoption weighs on aggregate productivity. In \citet*{aghion2019theory}, we develop a macroeconomic model that explains how a technological shock, such as that from information and communication technologies, can lead some companies to achieve a larger market size. However, this has simultaneously reduced the level of competition and, consequently, the incentive for firms to invest in innovation and new technologies, resulting in an overall slowdown.

The current slowdown is often described as ``a new Solow paradox'' because, although innovation appears to be everywhere, it is unclear if it is genuinely as pervasive as we perceive. To shed light on this, I have constructed \href{https://cverluise.github.io/patentcity/}{Patentcity}, a unique dataset detailed in \citet{bergeaud2022new}, which is derived from digitized patent documents from France, Germany, the UK, and the US spanning back to the $19^{th}$ century. This dataset meticulously charts the geography and nature of innovation over a lengthy period and provides additional insights into the primary drivers of innovation: firms and inventors. To relate innovation to long-term productivity growth, we propose a meticulous comparative analysis of the development of innovation in historical innovation hubs: the UK, France, Germany, the US and Japan in \citet*{ruveyda}. Finally, in \citet*{bergeaud2022rise} we focus on a more recent period and compare the dynamics of these countries with that of China in six illustrative technologies.

\section*{Misallocation, Public Policy and Innovation}

To gain a clearer understanding of the productivity slowdown, it is imperative to utilize individual data and approach the issue from a microeconomic perspective. The macroeconomic literature has clearly indicated that suboptimal policies (in areas such as competition, innovation, and industry) can have a considerable impact on productivity by steering resources away from their optimal use. In my second research avenue, I contribute to this body of knowledge by delving into the effects of different policies with a particular focus on the role of firms heterogeneity.

First, in \citet*{aghion2021impact} we show that inadequate labor market regulations can have a sizeable impact on firm innovation and result in a loss of growth and consumer welfare. To do so, we built a tractable and quantifiable endogenous growth model with size-contingent regulations and apply this to administrative firm panel data from France, where many labor regulations apply to firms with 50 or more employees. In another setting, we considered the impact of taxes that increase the adjustment cost of corporate real-estate in \citet*{bergeaud2021adjustment}. Specifically, we set and simulate a general equilibrium model with heterogeneous firms that predicts the response of firms to a productivity shock in the presence of fixed adjustment costs on real-estate. Using a large firm-level database merged with local real estate prices, we then exploit variations in the tax on capital gain to document a causal effect of adjustment costs on firms' labor demand and derive new results on the causes and implications of firms' local relocation. Finally, in \citet{aghion2019coase}, we show that a relaxation of banks' lending criteria could result in a negative aggregate effect on growth and innovation through a simple general equilibrium argument as better credit access allows less efficient incumbent firms to remain longer on the market, thereby discouraging entry of new and potentially more efficient innovators.

These three studies utilize both theoretical models and empirical analysis to highlight that policies which distort the allocation of firms' production resources can lead to significant adverse outcomes. These negative impacts arise primarily from a static misallocation of resources where labor, capital, and credit may not be channeled to the most suitable firms. In addition, these effects are magnified when considering the dynamic consequences of such misallocations through their impact of innovation and therefore growth. While these potential growth implications might be more profound in the long run, they are also considerably more challenging to predict.

This argument is further formalized in \citet*{aghion2022good}, where we consider the intricate relationship between static and dynamic misallocation and show how an environment with different types of firms of different comparative advantage will generally results in an inefficient allocation of resources. This inefficiency can be further magnified if the R\&D policy overlooks this diversity. A key policy takeaway from this research asserts that, to attain the social optimum, French research subsidies should predominantly support firms exhibiting high markups and generating positive externalities, what we term as ``good'' rent firms in our model. 

But how can we identify such firms? Pinpointing firms capable of generating positive spillovers via their innovations is elusive and challenging to predict. Nonetheless, in \citet*{bergeaud2022public}, our empirical findings indicate that R\&D policies bolstering the financing of high-quality public research institutions typically yield substantial positive spillovers, invigorating the innovation endeavors of some private firms. This result suggests that ``good firms'' are also these that can develop the capabilities to benefit from the production of new scientific discoveries in universities and public laboratories. In \citet*{aghion2022heterogeneous}, we highlight another evidence that firms are heterogeneous in their innovation response to global shocks. We use product level data from the customs authority to construct changes in demand. Using exhaustive data covering the French manufacturing sector, we show that French firms respond to exogenous growth shocks in their export destinations by patenting more; and that this response is entirely driven by the subset of initially more productive firms. This exposition to an increasing competition can therefore have differential impact if firms are not equipped to face it, either because of unsuitable institutions like in the case of Japan as we show in \citet*{respolito} or more generally because they are lagging behind in their innovation capabilities as in \citet*{aghion2021opposing}.

\section*{Transformation of the labor market and inequality}

My final research focus examines the implications of technology beyond its influence on growth and resource allocation, specifically its effects on the future of work and inequality. Both public debate and a large part of the scientific literature voice concerns about the potential effect on these questions. In my research, I underline a more nuanced view.

First, in \citet*{aghion2019innovation} we explore the connection between the rise in inequalities, particularly the wealth share captured by the very rich, and innovation in the United States since the 1970s. This article advocates for a dynamic perspective on inequalities. Indeed, while it is possible to demonstrate that innovation activities causally explain a significant part of the increase in inequalities, they have also contributed to enhancing social mobility, meaning the opportunity for all to prosper, while reducing the impact of family income. The rents created by innovation thus allow entrepreneurs to become wealthy quickly. However, if the right conditions are in place (proper education, healthy competition, availability of financing...), these rents will be temporary, allowing other entrepreneurs to benefit in turn. The central issue is understanding the origin of these rents and combating those stemming from an abusive dominant position of a company. This work offers a new perspective on the role of innovation and technology in the evolution of inequalities and provides a less pessimistic view than existing ones. In a subsequent study \citet*{abbg} based on British data, we delve deeper into the effects of innovation on the wage dynamics of unskilled workers. While they generally bear the brunt of technological advancement, especially with the acceleration of automation, the article highlights that certain non-cognitive abilities, that we called soft skills, remain in high demand by companies, even more so if these firms are technologically advanced. These soft skills can be acquired by individuals without formal education, and their significance prompts a reevaluation of educational models.

Of course, technological advancements will drive significant changes in both the labor market and the organization of firms. A natural recent instance is the surge in teleworking that emerged following the Covid-19 pandemic, and its subsequent effects on companies and their employees. The extent to which this new form or organization will affect workers and firms is still an open question, which I discussed in light of the macroeconomic literature in \citet*{bergeaud2020macroeconomics, bergeaud2023teletravail} and \citet*{bergeaud2023working}. But more generally, any shock that will impact the productivity or span of control of firms will have important effects for workers. For example, in \citet*{adsl} we show that a global communication and productivity shock, the arrival of broadband internet in France between 2000 and 2007 led firms to increase their outsourcing expenditure and decrease the diversity of occupations they employ in-house. This ``delayering'' of firms has important implication of workers and job satisfaction that I would like to investigate further in my future research, using newly available data on temporary work agency workers as in \citet*{interim} and on how workers value their job.


\newpage
\setstretch{1}
\bibliography{ref}
\bibliographystyle{chicago}

\end{document}


