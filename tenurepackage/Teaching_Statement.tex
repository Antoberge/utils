\documentclass[12pt]{article} 
\renewcommand{\baselinestretch}{1.3} 
\usepackage[round,authoryear]{natbib}
\usepackage[paper=a4paper,textheight=24cm,textwidth=17cm]{geometry}
\usepackage{amsmath, amstext, amssymb, dsfont}
\usepackage{textcomp}
\usepackage[sc]{mathpazo}
\usepackage[LGR,T1]{fontenc}
\usepackage{graphicx}
\usepackage{bbm, bm}
\usepackage{enumerate}
\usepackage{tabularx}
\usepackage{threeparttable}
\usepackage[scaled]{helvet} 
\usepackage{courier}
\newcolumntype{P}[1]{>{\centering\arraybackslash}p{#1}}
\usepackage{lscape}
\setlength{\parskip}{1ex plus 0.3ex minus 0.3ex} 
\setlength{\skip\footins}{1cm} 
\usepackage{enumitem}
 \usepackage[hang,flushmargin]{footmisc}
\usepackage{footnote}
\usepackage{multirow}
\usepackage{array}
\usepackage{lipsum}  
%\usepackage{xcolor}
\usepackage[usenames,dvipsnames]{color}
\usepackage{rotating}
\usepackage[utf8]{inputenc}
\usepackage{adjustbox}
\usepackage{datetime}
\usepackage{amsthm}
\setlength{\parindent}{0cm}
\usepackage{tikz}
\usepackage{fontawesome5}
\usepackage{relsize}
\usetikzlibrary{shapes.geometric, arrows, trees, positioning, fit, decorations.pathreplacing}
\definecolor{citecolor}{rgb}{.64,0,0}
\definecolor{bgreen}{rgb}{0.05, 0.5, 0.06}
\definecolor{bgfond}{RGB}{115, 106, 232}
\usepackage{longtable}
\usepackage[colorlinks=true, linkcolor=red, citecolor = citecolor, urlcolor = blue]{hyperref}
\usepackage{bookmark}
\usepackage{appendix}
\usepackage{tabularx}
\usepackage{booktabs}
\usepackage{caption}
\usepackage{siunitx}
\usepackage{setspace}
\usepackage{comment}
\usepackage{float}
\usepackage{subfig}
\usepackage[section]{placeins}

\begin{document}

\title{\textbf{Teaching Statement}}

\date{Antonin Bergeaud - Dec. 2023}

\maketitle


As a professor of economics, I deeply value the role of education in not just disseminating knowledge, but also in fostering intuition and cultivating curiosity. My teaching philosophy is built upon two foundational pillars: encouraging intuitive understanding and promoting deeper engagement.

\paragraph*{Intuitive Understanding} \phantom{;} \newline 
The subject I teach, macroeconomics and applied econometrics, are both vast and intricate. It is therefore easy for students to become overwhelmed by models, theories, and formulas. While these are vital components of theses disciplines, it is equally crucial for learners to grasp the underlying intuitions. In my classroom, I strive to present complex ideas in a way that connects with students' innate understanding. By anchoring abstract concepts to real-world examples and scenarios, I aim to illuminate the broader patterns and principles that drive macroeconomic behavior and give intuitions for students to avoid standard pitfalls in empirical analysis.

\paragraph*{Promoting Deeper Engagement} \phantom{;} \newline
My goal as an teacher extends beyond classroom lectures. I aspire to provide students with resources that allow them to delve deeper into topics that pique their interest. This is why I consistently offer additional references, blogs, books, and specially designed questions that can guide them in their exploration. By providing these materials, I hope to push students to be more active and to continue the exploration by themselves.

I see my students not just as learners but as potential future colleagues. This perspective shapes my interaction with them. I actively encourage students to challenge ideas presented in class, fostering an environment of mutual respect and intellectual growth. I believe that true understanding comes from debate, questioning, and discussion. To this end, I maintain an open-door policy, welcoming students to reach out to me, whether it is to discuss areas of dissatisfaction, delve further into specific topics, or simply to engage in a stimulating conversation about the fascinating world of macroeconomics. \\
\bigskip


\end{document}