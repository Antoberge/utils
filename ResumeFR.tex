\documentclass[12pt]{article} 
\renewcommand{\baselinestretch}{1.3} 
\usepackage[round,authoryear]{natbib}
\usepackage[paper=a4paper,textheight=24cm,textwidth=17cm]{geometry}
\usepackage{amsmath, amstext, amssymb, dsfont}
\usepackage{textcomp}
\usepackage[sc]{mathpazo}
\usepackage[LGR,T1]{fontenc}
\usepackage{graphicx}
\usepackage{bbm, bm}
\usepackage{enumerate}
\usepackage{tabularx}
\usepackage{threeparttable}
\usepackage[scaled]{helvet} 
\usepackage{courier}
\newcolumntype{P}[1]{>{\centering\arraybackslash}p{#1}}
\usepackage{lscape}
\setlength{\parskip}{1ex plus 0.3ex minus 0.3ex} 
\setlength{\skip\footins}{1cm} 
\usepackage{enumitem}
 \usepackage[hang,flushmargin]{footmisc}
\usepackage{footnote}
\usepackage{multirow}
\usepackage{array}
\usepackage{lipsum}  
%\usepackage{xcolor}
\usepackage[usenames,dvipsnames]{color}
\usepackage{rotating}
\usepackage[utf8]{inputenc}
\usepackage{adjustbox}
\usepackage{datetime}
\usepackage{amsthm}
\setlength{\parindent}{0cm}
\usepackage{tikz}
\usepackage{fontawesome5}
\usepackage{relsize}
\usetikzlibrary{shapes.geometric, arrows, trees, positioning, fit, decorations.pathreplacing}
\definecolor{citecolor}{rgb}{.64,0,0}
\definecolor{bgreen}{rgb}{0.05, 0.5, 0.06}
\definecolor{bgfond}{RGB}{115, 106, 232}
\usepackage{longtable}
\usepackage[colorlinks=true, linkcolor=red, citecolor = citecolor, urlcolor = blue]{hyperref}
\usepackage{bookmark}
\usepackage{appendix}
\usepackage{tabularx}
\usepackage{booktabs}
\usepackage{caption}
\usepackage{siunitx}
\usepackage{setspace}
\usepackage{comment}
\usepackage{float}
\usepackage{subfig}
\usepackage[section]{placeins}

\begin{document}

\title{\textbf{Resume}}
\date{Avril 2024}
\maketitle
\section*{Informations personnelles}

\begin{tabular}{r l r l }
    \textsc{Addresse :}   & 1 Rue de la Libération, 78350 Jouy-en-Josas  & \textsc{Nationalité:} & Française \\
    \textsc{Email:}     & bergeaud[at]hec.fr &     \textsc{Date de Naissance:} & 04/04/1989 \\ 
\end{tabular}

\section*{Affiliations actuelles}
\begin{tabular}{p{3cm}l}
\textsc{Depuis 2022} & Professeur associé d'économie, \textbf{HEC} \\
\textsc{Depuis 2022} & Senior Associate, \textbf{Programme on Innovation and Diffusion  (POID)} \\
\textsc{Depuis 2021} & Research Affiliate, \textbf{Centre for Economic Policy Research (CEPR)} \\
\textsc{Depuis 2019} & Associate, \textbf{Centre for Economic Performance (CEP), LSE} \\
\textsc{Depuis 2016} & Fellow of the Innovation Lab, \textbf{College de France} \\
\end{tabular}

\section*{Affiliations précédentes}

\begin{tabular}{p{3cm}l}
\textsc{2017-2022} & Economiste, \textbf{Banque de France} \\
\textsc{2020-2022} & Chargé d'enseignement, \textbf{Ecole Normale Supérieure de Paris} \\
\textsc{2018-2021} & Chargé d'enseignement, \textbf{SciencesPo Paris} \\
\textsc{2016-2020} & Visiting scholar, \textbf{Institute for Fiscal Studies}
\end{tabular}

\section*{Diplômes}
\begin{tabular}{p{2cm}l}	
 \textsc{2018} & Doctorat d'économie, \textbf{Paris School of Economics}\\
& \footnotesize{Superviseur: Prof. Philippe \textsc{Aghion}} \\
\textsc{2014} & M2 d'économie (Master APE), \textbf{Paris School of Economics}\\
\textsc{2014} & Diplome d'ingénieur statisticien, \textbf{ENSAE}\\
\textsc{2013} & Diplome d'ingénieur polytechnicien (X2010), \textbf{Ecole Polytechnique}\\
\end{tabular}

\section*{Recherche}

\subsection*{Publications}
\setstretch{1}
\begin{footnotesize}
    \begin{enumerate}
    \item  \emph{Bergeaud, A. et Guillouzouic, A.} ``Proximity of Firms to Scientific Production'' \textbf{\textcolor{red}{Annals of Economics and Statistics}}  (à paraitre) 
    \item \emph{Bergeaud, A. et Verluise, C.  (2023)}: ``Identifying technology clusters based on automated patent landscaping'' \textbf{\textcolor{red}{Plos One}} vol. 18(12)
    \item \emph{Bergeaud, A. et Verluise, C.  (2024)}: ``A new Dataset to Study a Century of Innovation in Europe and in the US''. \textbf{\textcolor{red}{Research Policy}} vol. 53(1), pages 1049-1153 (2024)
    \item \emph{Bergeaud A., Malgouyres, M., Mazet-Sonilhac, C. et Signorelli, S.}: ``Technological Change and Domestic Outsourcing''. \textbf{\textcolor{red}{Journal of Labor Economics}} (à paraitre)
    \item \emph{Aghion, P., Bergeaud, A. et Van Reenen, J. (2023)}: ``The Impact of Regulation on Innovation'' \textbf{\textcolor{red}{American Economic Review}} vol. 113(11), pages 2894-2936
    \item \emph{Aghion, P., Bergeaud, A., Lequien, M., Melitz, M. et Zuber, T. (2022)}: ``Opposing firm-level responses to the China shock: horizontal competition versus vertical relationships'' \textbf{\textcolor{red}{American Economic Journal: Economic Policy}} (à paraitre)
    \item \emph{Aghion, P., Bergeaud, A., Boppart, T., Klenow, P et Li, H. (2019)}: ``A Theory of Falling Growth and Rising Rents'' \textbf{\textcolor{red}{Review of Economics Studies}}, vol. 90(6), pages 2675–2702 
    \item \emph{Aghion, P., Bergeaud, A. Lequien, M. et Melitz, M. (2018)}: ``The Heterogeneous Impact of Market Size on Innovation: Evidence from French Firm-Level Exports'' \textbf{\textcolor{red}{Review of Economics and Statistics}} - (à paraitre)
    \item \emph{Bergeaud, A., Eymeoud, J.B., Garcia, T. et Henricot, D. (2023)}:``Working from Home and Corporate Real Estate'' \textbf{\textcolor{red}{Regional Science and Urban Economics}} - vol. 99(3), pages 1038-1078
    \item \emph{Ito, K., Bergeaud, A., Ikeuchi, K., Criscuolo, C. and Timmis, J. (2023)}: ``Global Value Chains et Domestic Innovation'' \textbf{\textcolor{red}{Research Policy}} - vol. 52(3), pages 1046-1099
    \item \emph{Bergeaud, A. et Ray, S. (2021)}: ``Adjustment Costs and Factor Demand: New Evidence From Firms' Real Estate'' \textbf{\textcolor{red}{Economic Journal}}, vol. 131(633), pages 70-100
    \item \emph{Bergeaud, A., Cette, G. et Lecat, R. (2020)}: ``Convergence of GDP per Capita in Advanced Countries over the 20th Century'' \textbf{\textcolor{red}{Empirical Economics}}, vol. 52, pages 2509–2526
    \item \emph{Aghion, P., Akcigit, A., Bergeaud, A., Blundell, R. et Hemous, D. (2019)}: ``Innovation and top income inequality,'' \textbf{\textcolor{red}{Review of Economic Studies}} Volume 86(1), pages 1–45
    \item \emph{Aghion, P., Bergeaud, A., Boppart, T., Klenow, P. et Li, H. (2019)}: ``Missing Growth from Creative Destruction,'' \textbf{\textcolor{red}{American Economic Review}}, vol. 109(8), pages 2795–2822
    \item \emph{Aghion, P., Bergeaud, A., Cette, G., Lecat, R. et Maghin, H. (2018)}: ``The Inverted-U Relationship Between Credit Access and Productivity Growth,'' \textbf{\textcolor{red}{Economica}}, vol. 86(341) pages 1-31
    \item \emph{Aghion, P., Bergeaud, A., Boppart, T., et Bunel, S. (2018)}: ``Firm Dynamics and Growth Measurement in France,'' \textbf{\textcolor{red}{Journal of the European Economic Association}}, vol. 16(4) pages 933–956
    \item \emph{Bergeaud, A., Cette, G. and Lecat, R. (2018)}: ``The role of production factor quality and technology diffusion in 20th century productivity growth,'' \textbf{\textcolor{red}{Cliometrica}}, vol. 12(1), pages 61-97. 
    \item \emph{Bergeaud, A., Cette, G. et Lecat, R. (2017)}: ``Total Factor Productivity in Advanced Countries: A Long-term Perspective,'' \textbf{\textcolor{red}{International Productivity Monitor}}, vol. 32(6) Centre for the Study of Living Standards
    \item \emph{Bergeaud, A., Potiron, Y. et Raimbault, J. (2017)}: ``Classifying patents based on their semantic contents,'' \textbf{\textcolor{red}{Plos One}} 12(4) 
    \item  \emph{Bergeaud, A., Cette, G. et Lecat, R. (2016)}: ``Productivity Trends in Advanced Countries between 1890 and 2012,'' \textbf{\textcolor{red}{Review of Income and Wealth}}, vol. 62(3), pages 420–444.  
\end{enumerate}
\end{footnotesize}

\subsection*{En révision}
\begin{footnotesize}
\begin{tabular}{p{1cm}p{14cm}}
1. & \emph{Bergeaud, A., Guillouzouic, A., Henry, E. et Malgouyres, C.}: ``From public to private: magnitude and channels of R\&D spillovers''. \textbf{\textcolor{red}{Quarterly Journal of Economics}}  \\
2. & \emph{Bergeaud, A., Schmidt, J. and Zago, R.}: ``Patents that match your standards: Firm-level Evidence on Competition and Innovation'' \textbf{\textcolor{red}{Journal of Financial Economics}}  \\
\cr

\end{tabular}
\end{footnotesize}
\subsection*{Documents de travail}
\begin{footnotesize}
\begin{tabular}{p{1cm}p{14cm}}
1. & \emph{Aghion, P., Bergeaud, A., Blundell, R. et Griffith, R.}: ``Social Skills and the Individual Wage Growth of Less Educated Workers'' \textbf{\textcolor{red}{CEPR Discussion Paper}} 14102 \\
\cr
2. & \emph{Bergeaud, A. et Verluise, C.}: ``The Rise of China's Technological Power: the Perspective from Breakthrough Technologies'' \textbf{\textcolor{red}{CEP Discussion Paper}} 1876 \\
\cr
3. & \emph{Aghion, P., Bergeaud, A., Gigout, M., Lequien, M. et Melitz, M.} ``Exporting ideas: How trade spills over to knowledge''\textbf{\textcolor{red}{CEP Discussion Paper 1960}}


\cr
\end{tabular}
\end{footnotesize}
\subsection*{Chapitre}
\begin{footnotesize}
\begin{tabular}{p{1cm}p{14cm}}
1. & \emph{Bergeaud, A., Cette, G. et Lecat, R.}: ``Productivity Growth and Real Interest Rates: A Circular Relationship'' in ``The Economics of Creative Destruction.
New Research on Themes from Aghion and Howitt'' edited by \emph{Ufuk Akcigit and John van Reenen}. \textbf{\textcolor{red}{Harvard University Press}} (2023)
\cr
\end{tabular}
\end{footnotesize}

\subsection*{Autres publications }
\begin{footnotesize}
    \begin{enumerate}
\item The impact of the COVID-19 pandemic and policy support on productivity  ECB Occasional Paper n.341 - 2024

\item Digitalisation: channels, impacts and implications for monetary policy in the euro area  ECB Occasional Paper n.266 - 2021

\item La conversion de l'immobilier de bureaux en immobilier résidentiel : quelles tendances après la Covid-19 et l'essor du télétravail ? Bulletin de la Banque de France, n°244 - 2023

\item Télétravail et productivité avant, pendant et après la pandémie de Covid-19  Economie et Statistique / Economics and Statistics, 539 

\item Difficultés de recrutement et caractéristiques des entreprises : une analyse sur données d'entreprises françaises Economie et Statistique / Economics and Statistics, 534-35, 43–59 

\item Réflexions sur la productivité: Après la pandémie et la guerre en Ukraine, quelles perspectives ? Revue Futuribles n°449 - 2022

\item Changement technologique et externalisation : illustration par l'accès à l'Internet haut débit en France Bulletin de la Banque de France, n°239 - 2022 

\item Dix ans après la réforme de la taxe professionnelle : quels effets sur le comportement des entreprises ? Bulletin de la Banque de France, n°238 - 2022 

\item Macroéconomie du Télétravail Bulletin de la Banque de France, Bulletin de la Banque de France, n°231 - 2020 

\item Covid-19 et chaînes de valeurs Bulletin de la Banque de France, n°230 - 2020 

\item Pouvoir de marché et croissance, quoi de neuf ? Bulletin de la Banque de France, n°226 - 2019 

\item Croissance de long terme et tendances de la productivité Revue de l'OFCE, (4), 43-62 - 2017

\item Le PIB par habitant et la productivité dans les économies avancées : regard sur le XXe siècle et perspectives pour le XXIe Bulletin de la Banque de France, n°211 - 2017.

\item Croissance économique et productivité. Un regard sur longue période dans les principales économies développées Revue Futuribles n° 417 - 2017.

\item Le produit intérieur brut par habitant sur longue période en France et dans les pays avancés : le rôle de la productivité et de l'emploi Revue d'économie et de statistique, numéro 474 - 2014.

\end{enumerate}
\end{footnotesize}

\subsection*{Conferences, séminaires et workshops}
\begin{footnotesize}
\begin{tabular}{p{1cm}p{13cm}}
2024 & Symposium in honor of Zwi Griliches (Paris, FR);  CoffeeNomics EU Comission (Brussels, BE); University of Porto (Porto, PT); Seminaire BETA (Strasbourg, FR); Utrecht University (Utrecht, NL); ECB Forum on Central Banking (Sintra, PT); LSE Workshop: Innovation and inequality in Europe and the USA (London, UK); VII MadMac Annual Conference (Madrid, ES); \\
2023 & HEC Entrepreneurship workshop (Paris, FR); NBER Summer Institute (Cambridge, USA); IOG-BFI Spring meeting (Paris, FR); Firm Organization and Capabilities workshop (Paris, FR); Entrepreneurship, Risk and Talent workshop (Paris, FR); Unversity of Amsterdam (Amsterdam, NL); PSE Global Issue Conference (Paris, FR); Conférence Innovation at UQAM (Montreal, CA); Warwick-CFM-Vienna Macro Conference (London, UK); Glasgow University (Glasgow, UK); Paris 8 (Paris, Fr) \\
2022 & University of Oslo (Oslo, Norway); Hitotsubashi University (virtual); POID-CEP brownbag (virtual); BSE Summer Forum (Barcelona, Spain); Journées Louis-André Gérard-Varet (Marseille, France); Université de Cergy (Paris, France); University of Nottingham (Nottingham, UK); NBER Summer Institute (Cambridge, USA); Séminaire GAINS (Le Mans, France) \\ 
 2020-2021 &  ULB department seminar (Brussels, Belgium); KU Leuven-Google conference (virtual);  Royal Economic Society (virtual); Banca d'italia (virtual); CREST (virtual); College de France (virtual); HEC (Paris, France); IZA worshop on labor market institutions (virtual); Journées de Microéconomie Appliquée (virtual); RCEA seminar (virtual); ESCB research cluster (virtual) \\
 2019 & AEA annual meeting (Atlanta, USA); Ghent university (Ghent, Belgium); Journées Louis-André Gérard-Varet (Aix, France); Toulouse Business School economic workshop (Toulouse, France); EPFL's Workshop on Computational Methods in Social Science (Lausanne, Switzerland); Royal Economic Society Annual Meeting (Warwick, UK); CREST (Paris, France); WGEM (Paris, France); Univ of Bath (Bath, UK); National Tax Administration Annual Meeting (Tampa, US); College de France (Paris, France); BdF-BdI-SciencesPo Conference (Rome, Italy) \\ 
 2018 &  Royal Economic Society Annual Meeting (Brighton, UK); Banca de Italia-CEPR (Roma, Italy); Journées Louis-André Gérard-Varet (Aix, France); RCEA (Rimini, Italy); OECD Global Forum in Productivy (Ottawa, Canada); Banque de France (Paris, France); Young Association (London, UK)  Forschungskolloquium universitat Giessen (Giessen, Germany);  CFE (Pisa, Italy); ECB/CEPR Labour Market Workshop (Frankfurt, Germany); G7 digitalization workshop (Ottawa, Canada)\\
 2017 & Banco de Espana (Madrid, Spain); NBER Summer Institute (Cambridge, USA); Association Française de Science Economique (Nice, France); Banque de France-AMSE (Paris, France); Journées Louis-André Gérard-Varet (Aix, France);\\
 Before 2016 & Association Française de Science Economique (Nancy, France and Rennes, France); Journées Louis-André Gérard-Varet (Aix, France) 
 \end{tabular}
\end{footnotesize}

\subsection*{Activités éditoriales}

Depuis 2016, j'ai évalué 60 articles pour les journaux suivants : 

\begin{footnotesize}
\begin{tabular}{p{1cm}p{13cm}}
& American Economic Review; Econometrica; Quarterly Journal of Economics; Journal of Political Economics; American Economic Journal: Macroeconomic; American Economic Review Insight; Economic Journal; Journal of European Economic Association; International Economic Review; Research Policy; Journal of International Economics; Review of Economic Dynamics; Economic Modelling; Quantitative Economics; European Economic Review; Regional Science and Urban Economics; Labour Economics; Canadian Journal of Economics; Economica; PLOS ONE; Journal of Industrial Organization; Economics of Transition; Political Studies;  International Productivity Monitor; Economic Bulletin; Revue Economique \\
\end{tabular}
\end{footnotesize}

\subsection*{Prix et bourses}

\begin{itemize}
    \item Fondation HEC - bourse F-budget (2024) - 25000 EUR pour le projet Laboratoire du futur du travail
    \item HEC Qatar grant (2023) - 5000 euros pour développer une seconde version de PatentCity
    \item Google for Education grant (2020) - 5000 USD en crédit cloud pour le project PatentCity
    \item Prix Turgot de la pédagogie économique (2019) pour ``Le Bel Avenir de la Croissance'' (Odile Jacob, 2018) 
    
\end{itemize}
\end{document}
